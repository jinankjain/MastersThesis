% Chapter Introduction

\chapter{Introduction} % Main chapter title

\label{intro} % For referencing the chapter elsewhere, use \ref{intro}

\section{Motivation}
During the early days of the Internet, there was an implicit trust among the users of the Internet. However, nowadays the Internet operates in a different environment. Today, communication often takes place between parties that do not necessarily trust each other. Communication patterns have evolved: previously it was only used for some research purpose but now it handles critical activities such as bank transactions. Moreover, some parties engage in malicious activities, such as attacking other hosts or even the network itself, e.g., for personal gains or due to conflicts of interest.

The drastic transition on how the Internet is used led to discussions regarding how to make the Internet more "secure". For example, some argue for source accountability guarantees to hold the source responsible for any traffic that it originates, while others argue for functionalities to enhance user privacy.

\subsection{Importance of Source Accountability}
It is very important for users of the Internet to have a sense of trust i.e., they want assurance that they are communicating with whom they think they are communicating. Additionally, lack of source accountability is the source of lots of Denial Service Attacks on today's the Internet. Attackers spoof their victim's addresses and launch reflection attacks exhausting their victim's resources.

\subsection{Importance of Communication Privacy}
Privacy is a limit on government power, as well as the power of private sector companies. The more someone knows about us, the more power he or she can have over us. Personal data is used to make very important decisions in our lives. Personal data can be used to affect our reputations, and it can be used to influence our decisions and shape our behavior. It can be used as a tool to exercise control over us. And in the wrong hands, personal data can be used to cause us great harm.

These days everyone is aware of mass-surveillance sponsored by governments and his or her capabilities are rapidly advancing with big data and data mining technologies he or she can collect even larger volumes of data, and analyze the data more thoroughly. Even private companies such as Facebook, Google etc are using this immense to manipulate user choices and perform more target advertising.

Despite the importance, today's Internet does not provide any sort of privacy guarantees and users are forced to use complicated tools such as Tor and VPN services. Such tools are either cumbersome, degrade performance, or provide only weak security guarantees. 

Both source accountability and communication privacy are important properties but today's the Internet fails to provide either of them. For example, an IP address is insufficient to be an identifier since it can be spoofed while at the same time it reveals too much information about the users revealing their true identities.

Even if the Internet is modified to support both properties, then also supporting both properties at the same time would be challenging goals because of the inherent conflicting nature of those problems. Source accountability requires identification of the sender, but to guarantee privacy the sender's identity should not be revealed.

\section{Problem Definition} \label{intro:prob}
Our objective is to design a network protocol in which source accountability while preserving communication privacy is a first class citizen. This section tries to formalize these seemingly conflicting goals and the security properties we strive to achieve with this new protocol.

\subsection{Source Accountability}
Source accountability refers to a link between the identity of sender host and sent packet which is impossible to counterfeit. Thus it equates to the property of non-repudiation i.e., the sender cannot deny having sent a packet nor he can be falsely accused of sending a packet which it did not send.

In order to achieve source accountability, it boils down two fundamental requirements. First, we must establish a strong notion of host identity to prevent a malicious host from creating multiple identities or impersonating other hosts in the network. Second, we must create a link between a host identity and the packet which it sends on the network. The link must be vetted by some third party (e.g the source AS) that is not the sender itself since senders themselves can be malicious. To this extent, the third party must observe all of the sender’s traffic such that every packet in the network can be linked to a specific sender.
\newline \newline
\textbf{Adversary Model} \newline \newline
The objective is to break the source accountability by creating a legitimate packet on behalf of someone else in the network. We assume that the adversary can inhabit in multiples ASes and that he/she can inspect all packets within those ASes. Specifically, the adversary can eavesdrop on all control and data messages in the network, but cannot compromise the secret keys of the ASes that it resides in.

\subsection{Communication Privacy}
The first goal with respect to privacy at the network layer is the host privacy. To accomplish the host privacy, the identity of the host must be hidden from any other host in the source AS that is not in the same broadcast domain as the host t (e.g., on the same WiFi network or LAN segment), any transit network that forwards traffic, as well as the destination AS (including the communication peer). The host cannot hide its identity from the source AS since the AS knows all the information about its customers as well as their network attachment points. A host cannot hide from other hosts on the same broadcast domain since the layer 2 address is visible. That is the reason we address the host privacy at the network layer i.e., network layers headers should not disclose host identity. It is totally possible that higher level protocols (e.g. HTTP cookie) might still leak host identity; however, these aspects are out of scope for this work.

In addition, our notion of host privacy includes \textit{sender-X} unlinkability, where 'X' could be a flow, an application, etc. We describe a concrete definition of sender-X unlinkability where ‘X’ is a flow to provide an intuition about sender-X unlinkability. Sender-flow unlinkability means that simply by observing packet contents (both headers and payloads) of any number of flows originating from the same AS, the source(s) of the flows are no more and no less related after the observation than they were before the observation.

Our second goal is data-privacy by providing end-to-end encryption at the network layer. Data should never fall in hands of unintended recipients, including the source and the destination ASes. To this end, the protocol should natively (i.e., without relying on upper-layer protocols, such as TLS \cite{tls}) support the secure key establishment that would create a shared symmetric key to encrypt the communication and that is resilient against man-in-the-middle (MitM) attack.

Moreover our notion of data privacy also includes perfect forward secrecy (PFS). PFS is a feature of specific key agreement protocols that gives assurances your session keys will not be compromised even if the private key of the server is compromised. PFS protects past sessions against future compromises of secret keys or passwords. By generating a unique session key for every session that a user initiates, a compromise of a single session key will not affect any other data than the data exchanged in that specific session of the compromised key. Thus disclosure of long-term secret keying material does not compromise the secrecy of exchanged keys from past sessions, and thus data privacy of prior communication sessions is guaranteed.
\newline \newline
\textbf{Adversary Model} \newline \newline
The goal of attacker is to compromise the host privacy by determining the identity of the host or can determine if two ‘X’s from the same source AS originate from the same host. We assume that the adversary can  control any entity in the Internet except for the source host, hosts that are in the same broadcast domain as the source host, and the source AS itself. The adversary can observe packet headers and content, but we do not consider timing analysis techniques, such as inter-packet arrival times.
\\ \\
We argue that architecture should only provide the basic building blocks to achieve the host privacy at the network layer; stronger privacy guarantees (e.g. resiliency against the timing side channels attacks) should be taken care by higher level protocols (e.g., transport protocol). . For instance, a transport protocol could implement a packet scheduling algorithm that alters timing between packets to mitigate traffic identification based on inter-packet timing analysis. Our argument against providing strong privacy guarantees is that it comes with a performance cost and not every application or host requires such strong privacy guarantees. . Hence, protocols that offer stronger privacy guarantees are left to upper layers so that users can choose the appropriate protocol based on their requirements.

In order to violate data privacy adversary can decrypt the communication content exchanged between two hosts. To this end, we assume that the adversary can control any entity in the Internet except for the two communicating hosts, and one of the two ASes that the hosts reside in.

\section{Summary}
APNA \cite{taeho_thesis, source_accountability} describes the key idea about how we can achieve a sweet spot between accountability and privacy based on the problem definition in Section \ref{intro:prob}. But it still fails to describe all the necessary services which are required to have a working implementation of APNA protocol. There are multiple instance where the services are under-specified.

Furthermore, there is no real implementation of the architecture and hence it has never been deployed on any future the Internet architecture testbed. Consequently, the past work \cite{taeho_thesis, source_accountability, aip, apip} did not offer any kind of end-to-end analysis such as latency and bandwidth experiments and assess the overhead due to the privacy enhancements of the APNA protocol. In summary, the following high level contributions are made in this thesis.

\begin{itemize}
    \item All the micro-services (i.e., EphID Mgmt. service, DNS service, Identity Management service and Key Management service) are redesigned and implemented in an efficient way as an APNA Management Service. An extensive microbenchmarking has been performed on all these services to evaluate their response time under heavy load.
    \item Since there are multiple approaches to integrate APNA inside SCION we explore three different ways for the integration. We discuss the advantages and disadvantages of each approach in detail.
    \item We deploy one of the approaches on the testbed for SCION (i.e., SCIONLab). Then, we perform some performance measurements related to bandwidth and latency to assess the overhead of the APNA protocol over vanilla SCION.
\end{itemize}