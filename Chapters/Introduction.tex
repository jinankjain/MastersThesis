% Chapter Introduction

\chapter{Introduction} % Main chapter title

\label{intro} % For referencing the chapter elsewhere, use \ref{intro}

\section{Motivation}
During the early days of Internet there was an implicit trust among the users of the Internet. However Internet has just exploded and gone are those days when you could trust anyone on the Internet. Today, communication often takes place between parties that do not necessarily trust each other. Communication patterns have evolved previously it was only used for some research stuff but now it handles something as serious as bank transactions. Moreover, some parties engage in malicious activities, such as attacking other hosts or even the network itself, e.g., for personal gains or due to conflicts of interest.

Due to this drastic transition it has sparked some serious conversation like source accountability guarantees to hold the source responsible for any traffic that it originates, and whether the network should provide functionalities to enhance user privacy.

\subsection{Importance of Source Accountability}
Its very important for users of Internet to have a sense of trust i.e., they want assurance that they are communicating with whom they think they are communicating. Additionally lack of of source accountability is the source of lots of Denial Service Attacks on today's Internet. Attackers spoof their victim's addresses and launch reflection attacks exhausting their victim's resources.

\subsection{Importance of Communication Privacy}

\section{Problem Definition}
Our objective is to design a network protocol in which source accountability while preserving communication privacy is a first class citizen. This section tries to formalize these seemingly conflicting goals and the security  properties we strive to achieve with this new protocol.

\subsection{Source Accountability}
Source accountability refers to a link between identity of sender host and sent packet which is impossible to counterfeit. Thus it property of non-repudiation i.e., the sender cannot deny having sent a packet nor he can be falsely accused of sending a packet which it did not sent.

In order to achieve source accountability it boils down two fundamental requirements. First a strong notion of host identity so the host cannot create multiple identities nor impersonates another hosts in the network. Second a link between a host identity and the packet which it sends on the network. The link must be vetted by some third party (e.g the source AS) that is not the sender itself, since senders themselves can be malicious. To this extend, the third party must observe all of the sender’s traffic such that every packet in the network can be linked to a specific sender.

\subsubsection{Adversary Model} 

The objective is to break the source accountability by creating a legitimate packet on behalf of someone else in the network. We assume that the adversary can inhabit in multiples ASes and that he/she can inspect all packets within those ASes. Specifically, the adversary can eavesdrop on all control and data messages in the network, but cannot compromise the secret keys of the ASes that it resides in.

\subsection{Communication Privacy}
The first goal with respect to privacy at the network layer is the host privacy. To accomplish the host privacy, the identity of the host must be hidden from any other host in the source AS that is not in the same broadcast domain as the host t (e.g., on the same WiFi network or LAN segment), any transit network that forwards traffic, as well as the destination AS (including the communication peer). The host cannot hide its identity from the source AS, since the AS knows all the information about its customers as well as their network attachment points. A host cannot hide from other host on the same broadcast domain, since the layer 2 address is visible. That is the reason we address the host privacy at the network layer i.e., network layers headers should not disclose host identity. It is totally possible that higher level protocols (e.g. HTTP cookie) might still leak host identity; however these aspects are out of scope for this work.

In addition our notion of host privacy includes \textit{sender-X} unlinkability, where 'X' could be a flow, an application, etc. We describe a concrete definition of sender-X unlinkability where ‘X’ is a flow to provide an intuition about sender-X unlinkability. Sender-flow unlinkability means that simply by observing packet contents (both headers and payloads) of any number of flows originating from the same AS, the source(s) of the flows are no more and no less related after the observation than they were before the observation.

Our second goal is for data-privacy by providing end-to-end encryption at the network layer. Data should never fall in hands of unintended recipients, including the source and the destination ASes. To this end, protocol should natively (i.e., without relying on upper-layer protocols, such as TLS) support establishing secure communication channel between hosts by negotiating a shared secret key and protect against Mann-in-the-Middle (MitM) attacks.

Moreover our notion of data privacy also includes perfect forward secrecy (PFS). PFS is a feature of specific key agreement protocols that gives assurances your session keys will not be compromised even if the private key of the server is compromised. PFS protects past sessions against future compromises of secret keys or passwords. By generating a unique session key for every session a user initiates, even the compromise of a single session key will not affect any data other than that exchanged in the specific session protected by that particular key. Thus disclosure of long-term secret keying material does not compromise the secrecy of exchanged keys from past sessions, and thus data privacy of prior communication sessions is guaranteed.

\subsubsection{Adversary Model} 

The goal of attacker is to compromise the host privacy by determining the identity of the host or can determine if two ‘X’s from the same source AS originate from the same host. We assume that the adversary can  control any entity in the Internet except for the source host, hosts that are in the same broadcast domain as the source host, and the source AS itself. The adversary can observe packet headers and content, but we do not consider timing analysis techniques, such as inter-packet arrival times.
\\ \\
We argue that architecture should only provide the basic building blocks to achieve the host privacy at the network layer; stronger privacy guarantees (e.g. resiliency against the timing side channels attacks) should be taken care by higher level protocols (e.g., transport protocol). . For instance, a transport protocol could implement a packet scheduling algorithm that alters timing between packets to mitigate traffic identification based on inter-packet timing analysis. Our argument against providing strong privacy guarantees is that it comes with a performance cost and not every application or host requires such strong privacy guarantees. . Hence, protocols that offer stronger privacy guarantees are left to upper layers so that users can choose the appropriate protocol based on their requirements.

In order to violate data privacy adversary can decrypt the communication content exchanged between two hosts. To this end, we assume that the adversary can control any entity in the Internet except for the two communicating hosts, and one of the two ASes that the hosts reside in.