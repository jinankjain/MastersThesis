% Chapter Overview

\chapter{Overview} % Main chapter title

\label{overview} % For referencing the chapter elsewhere, use \ref{Chapter3}

\section{Different components involved in SCION}
\begin{itemize}
\item Border Router
\item Dispatcher
\item SCIOND
\end{itemize}

\section{APNA as native socket address family i.e AF\_APNA}
\subsection{Modifications}
\begin{itemize}
	\item Border Router
    \item Dispatcher
    \item SNET
\end{itemize}

\subsection{Communication Overview}

\subsection{Deployment on SCIONLab}
This approach requires significant modification in the scion lab infrastructure as all the border routers needs to understand this new address family for packet parsing and forwarding the packet

\section{Implement APNA as SCION Service}
\subsection{Modifications}
\begin{itemize}
	\item SCIOND both on python and golang side
    \item Topology generator
    \item SNET
\end{itemize}

\subsection{Communication Overview}

\subsection{Deployment on SCIONLab}
This approach is deployable without significant modifications just endhost ASes need to be modified so that SCIOND could resolve this new APNA SVC address and rest of forwarding is taken care by APNA SVC

\section{Implement APNA completely agnostic of network inside Data packet}
\subsection{Modification}
\begin{itemize}
\item Border Router
\end{itemize}

\subsection{Communication Overview}

\subsection{Deployment on SCIONLab}
This approach requires modification only to endhost border router as they need to perform mac verification before forwarding the pkt and also decipher hostID from EphID at the last Hop.

\section{Apna Management Service}

\subsection{Issue Ephemeral IDs}
\subsection{Mock DNS}
\subsection{Service to obtain pub key with which pkt was signed}
\subsection{Conversion between HostIDs and IP address}

\section{APNA Packet Description}

\section{}
